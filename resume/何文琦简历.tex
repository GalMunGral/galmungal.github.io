\documentclass[10 pt]{article}
\usepackage{xeCJK}
\usepackage[margin=0.5in]{geometry}
\usepackage{titlesec}
\usepackage{enumitem}
\usepackage[T1]{fontenc}

\titleformat{\section}{}{\thesection}{0pt}{\vspace{-0.5em}}[{\titlerule[0.1pt]}]
\setCJKmainfont{Noto Serif CJK SC}
\setlist[itemize]{leftmargin=1.5em, topsep=0.5em, itemsep=0pt, label={-}}
\setlength\parindent{0pt}
\pagestyle{empty}

\begin{document}

\begin{center}
\textbf{\huge 何文琦}\\
\vspace{0.2em}
\textbf{\texttt{hewenqi96@\{gmail,qq\}.com | 136-2602-2079 }}
\end{center}

\section*{教育经历}
\textbf{计算机科学理学学士 | 佐治亚理工学院 (Georgia Institute of Technology)} | GPA: 3.97/4.0 \hfill 2015/08 -- 2019/12

\section*{工作/实习经历}
\textbf{前端工程师 | 谷露软件 Gllue Software | 上海} \hfill 2020/11 -- present
\begin{itemize}
\item 负责谷露星选小程序猎头榜单以及登录注册和团队管理模块的前端开发.
\item 主要使用 TypeScript, Taro, React, dva (Redux + Redux-Saga + Immer), Linaria (styled-components), Sass, ECharts.
\item 参与小程序前端基础建设, 包括表单引擎, 基于 react-pdf 的海报图片生成服务, 以及可嵌套组合的数据查询语言.
\item 参与了 ATS 和其中 XML 渲染引擎的维护和开发. 编写了检测无用代码文件的 webpack 插件.
\end{itemize}

\textbf{Electron 开发工程师 (兼职) | Étude (\texttt{etudereader.com}) | 亚特兰大} \hfill 2019/09 -- 2019/12
\begin{itemize}
\item 应用 MVVM 模式和插入区间的算法实现了 PDF 阅读器非连续文本选中和高亮显示的功能.
\item 使用正则表达式从目录页文本中提取标题和页码, 实现了自动生成章节跳转链接的功能.
\end{itemize}

\textbf{前端开发实习生 | PegasusCRM | 亚特兰大} \hfill 2018/01 -- 2018/04
\begin{itemize}
\item 主要使用 Laravel (PHP), Vue.js, jQuery 和 Sass.
\end{itemize} 

\section*{个人/学校项目 \texttt{\large //github.com/GalMunGral}}
\textbf{ \texttt{./modulizer:} 独立开发的前端构建工具 (受Vite和webpack启发) } \hfill 2020/10 -- 2020/11 
\begin{itemize}
\item 独立开发了基于 Babel, PostCSS, Puppeteer, 支持打包和非打包模式的前端构建工具和开发服务器.
\item (原生 ESM 模式) 支持构建缓存, 自动刷新 (构建缓存失效+页面刷新), HTTP/2 (服务器推送功能暂未完成).
\item (打包+服务端渲染模式) 设计并实现了代码分割 算法 (同步/异步/公用/第三方模块)和模块异步加载机制.
\end{itemize}

\textbf{ \texttt{./replay:} 独立开发的前端框架 (受 React Fiber, MobX 和 Vue 启发) } \hfill 2020/05 -- 2020/09
\begin{itemize}
\item 设计并分别通过两种渲染模式实现了diff 算法. 基于 proxy 实现了独立于渲染机制的响应式系统.
\item (增量DOM模式) 将 JSX 直接编译为原地 diff 指令, 减少内存消耗. 增加 hydration 功能, 支持服务端渲染.
\item (虚拟DOM模式) 用 generator 实现了可暂停/继续/取消并支持组件懒加载的异步渲染. 支持一套原创的内嵌DSL.
\item 在上述核心库的基础上开发了简单的和 CSS-in-JS 库 (仿 styled-components), 状态管理 (仿 Redux, Vuex) 和前端路由.
\item 使用本框架搭建了仿 Gmail 的单页应用, 用于对本框架和上述自行开发的构建工具进行测试和调试.
\end{itemize}

\textbf{\texttt{./js-to-wasm:}简易 JavaScript -- WebAssembly 编译器 (仅支持小部分语法)} \hfill 2020/06
\begin{itemize}
\item 用 Node.js 开发了美观输出 WebAssembly 文本格式的 JS 编译器, 内部 AST 大致遵循 ESTree 规范.
\end{itemize}

\textbf{\texttt{./marta-\{react,vue\}:} 地铁乘客信息管理系统 (SQL课程项目) } \hfill 2017/09 -- 2017/11
\begin{itemize}
\item 分别使用 React/Redux/React Router 和 Vue/Vuex/Vue Router 搭配 Bulma  CSS 框架开发了两套前端应用.
\item 使用 MySQL, Express 和 JWT 搭建了后端 API, 提供记录行程, 管理账户余额, 查看历史记录和车站人流等功能.
\item 用 Node.js 开发了通过 cookie 指定上游端口的简易反向代理, 满足了在同一个 Heroku 项目部署多个应用的需求.
\end{itemize}

\textbf{\texttt{../MarcusWilder/open-evaluation:} 期中调查问卷系统 (设计课程项目)} \hfill 2019/08 -- 2019/12
\begin{itemize}
\item 参与 Angular 前端开发, 实现了将 JSON 格式的问卷模板和用户回答信息动态渲染成表单的功能.
\item 负责 Express/Node.js 后端开发, 集成了学校单点登录系统 (CAS), 并通过 Canvas LMS API 拉取学生选课数据.
\item 利用 Puppeteer 截取 session cookie 作为替代 OAuth 2.0 临时获取 API 访问权限的方法 (未公开, 仅作 demo 用).
\end{itemize}

\textbf{\texttt{./web-repl:} 在线 Python REPL 聊天室}  \hfill 2018/05 
\begin{itemize}
\item 用 Node.js 和 Socket.IO 开发了基于动态创建 Python 子进程的可协作运行代码的简易聊天室 (前端以 React 编写).
\item 用 Socket.IO 搭建了 WebRTC 视频通话功能所需的信令服务. (另用 Node.js 编写了 WebSocket 服务器作为替代).
\end{itemize}

\textbf{\texttt{./sitbit-\{ios,android,server\}:} 移动端久坐追踪应用} \hfill 2019/01
\begin{itemize}
\item 使用 iOS Core Motion 和安卓 sensor 框架实时监测用户起身/坐下的动作并上传到数据库 (后端以 Go 语言编写).
\item 使用 D3.js 以类似 GitHub 个人页上的 SVG 日历热图的形式展示每日久坐时间.
\end{itemize}

\end{document}

