\documentclass[10 pt]{article}
\usepackage{xeCJK}
\usepackage[margin=0.5in]{geometry}
\usepackage{titlesec}
\usepackage{enumitem}
\usepackage[T1]{fontenc}

\titleformat{\section}{}{\thesection}{0pt}{\vspace{0pt}}[{\titlerule[0.1pt]}]
\setCJKmainfont{Noto Serif CJK SC}
\setlist[itemize]{leftmargin=1.5em, topsep=0.5em, itemsep=0pt, label={-}}
\setlength\parindent{0pt}
\pagestyle{empty}

\begin{document}

\begin{center}
\textbf{\huge 何文琦}\\
\vspace{0.2em}
\textbf{\texttt{hewenqi96@qq.com | 136-2602-2079 }}
\end{center}

\section*{教育经历}
\textbf{计算机科学理学学士 | 佐治亚理工学院 (Georgia Institute of Technology)} | GPA: 3.97/4.0 \hfill 2015/08 -- 2019/12

\section*{工作/实习经历}
\textbf{前端工程师 | 谷露软件 Gllue Software | 上海} \hfill 2020/11 -- present
\begin{itemize}
\item 负责协作平台人才库模块的前端开发, 主要使用 TypeScript, React, SWR, Tailwind CSS, Vite.
\item 负责谷露星选小程序多个模块的前端开发. 主要使用 TypeScript, Taro (React), dva (Redux Saga), Linaria, Sass, ECharts.
\item 参与小程序前端基础建设, 包括表单引擎、图片生成服务.
\item 参与了 ATS 和其中 XML 渲染引擎的维护和开发.
\end{itemize}

\textbf{Electron 开发工程师 (兼职) | Étude (\texttt{etudereader.com}) | 亚特兰大} \hfill 2019/09 -- 2019/12
\begin{itemize}
\item 使用正则表达式从目录页文本中提取标题和页码, 实现了自动生成章节跳转链接的功能.
\end{itemize}

\textbf{前端开发实习生 | PegasusCRM | 亚特兰大} \hfill 2018/01 -- 2018/04
\begin{itemize}
\item 主要使用 Laravel (PHP), Vue.js, jQuery 和 Sass.
\end{itemize} 

\section*{个人项目 \texttt{\large //github.com/GalMunGral}}

\textbf{\texttt{./parser-combinator:}  parsec (Haskell) 式解析器组合子的 JS 实现  (概念验证)} \hfill 2021/05
\begin{itemize}
\item 实现了应用式函子和单子两种组合方式. 编写了简易的 JSON, XML, JavaScript (小部分语法) 解析器作为验证.
\end{itemize}

\textbf{\texttt{./react-split-renderer}  基于 Service Worker 的 React 渲染器 (概念验证)} \hfill 2021/05 
\begin{itemize}
\item  在 worker 内维护一层数据结构作为主线程 DOM 的镜像, 通过线程间的消息机制代理 React Reconciler 的操作.
\item 实现了在后台线程运行前端应用, 理论上可以减少重复加载执行初始化代码 (比如同时打开多个 tab 时) 的消耗.
\end{itemize}

\textbf{\texttt{./\{atomic-css, unused-files, coupling-analyzer\}-webpack-plugin:} } \hfill 2020/04
\begin{itemize}
\item 一些 webpack 插件, 实现了包括按需生成原子化CSS、检测无用代码文件、分析模块间耦合等功能.
\end{itemize}

\textbf{\texttt{./modulizer:} 受 Vite 和 webpack 启发的前端构建工具} \hfill 2020/09 -- 2020/10
\begin{itemize}
\item 基于 Babel 和 PostCSS 实现了基于闭包的打包模式和 ESM + HTTP/2 模式 (默认开启构建缓存, 自动刷新).
\item 实现了打包模式的代码分割 (同步/异步/公用/第三方模块) 和模块动态导入机制.
\end{itemize}

\textbf{\texttt{./replay:} 受 React/MobX 和 Vue 启发的前端框架} \hfill 2020/05 -- 2020/09
\begin{itemize}
\item 编写 Babel 插件实现了基于 JS 自身语法的 DSL, 用 generator 实现了可中断/取消的非阻塞渲染 (diff), 支持组件懒加载.
\item 设计并实现了增量DOM模式, 将 DSL 直接编译为 (原地) diff 指令, 减少虚拟 DOM 节点的生成和销毁.
\item 实现了 hydration、(依赖收集式) 响应式机制、状态管理、前端路由、CSS-in-JS, 并编写了仿 Gmail 的单页应用作为验证.
\end{itemize}


\textbf{\texttt{./marta-\{react,vue\}:} 地铁乘客信息管理系统 (SQL课程项目) } \hfill 2017/09 -- 2017/11
\begin{itemize}
\item 主要实现了记录行程, 管理账户余额, 查看历史记录和车站人流等功能.
\item 使用了 React/Redux/React Router, Vue/Vuex/Vue Router, Bulma  CSS, MySQL, Sequelize, Express, jsonwebtoken.
\item 为了将多个应用部署到同一个 Heroku 项目, 用 Node.js 实现了简易的反向代理 (通过 cookie 选择上游端口).
\end{itemize}

\textbf{\texttt{../MarcusWilder/open-evaluation:} 期中调查问卷系统 (设计课程项目)} \hfill 2019/08 -- 2019/12
\begin{itemize}
\item 参与 Angular 前端开发, 主要实现了将 JSON 格式的问卷模板和用户填写的数据动态渲染成表单的功能.
\item 负责 Express/Node.js 后端开发, 集成了学校 CAS 单点登录系统, 并通过 Canvas LMS API 拉取学生选课数据.
\item 利用 Puppeteer 模拟用户登录并截取 session cookie 作为替代 OAuth 2.0 临时获取 API 访问权限的方案.
\end{itemize}

\textbf{\texttt{./web-repl:} 在线 Python REPL 聊天室}  \hfill 2018/05 
\begin{itemize}
\item 开发了基于动态创建 REPL 子进程的简易聊天室, 可协作运行代码. 主要使用 Node.js, Socket.IO, WebRTC, React
\end{itemize}

\end{document}

