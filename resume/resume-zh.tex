\documentclass[10 pt]{article}
\usepackage{xeCJK}
\usepackage[margin=0.5in]{geometry}
\usepackage{titlesec}
\usepackage{enumitem}
\usepackage[T1]{fontenc}

\titleformat{\section}{}{\thesection}{0pt}{\vspace{0pt}}[{\titlerule[0.2pt]}]
\setCJKmainfont{Noto Serif CJK SC}
\setlist[itemize]{leftmargin=1.5em, topsep=1em, itemsep=0.5pt, label={-}}
\setlength\parindent{0pt}
\pagestyle{empty}

\begin{document}

\begin{center}
\textbf{\huge 何文琦}\\
\vspace{1em}
\texttt{hewenqi96@qq.com | 136-2602-2079 }
\end{center}

\section*{教育经历}
\textbf{计算机科学理学学士 | 佐治亚理工学院 (Georgia Institute of Technology)} | GPA: 3.97/4.0 \hfill 2015/08 -- 2019/12

\section*{工作经历}
\textbf{兼职 JavaScript 开发工程师 | Etude (\texttt{etudereader.com}, 大学生初创公司) | 亚特兰大} \hfill 2019/09 -- 2019/12
\begin{itemize}
\item 应用 MVVM 模式和插入区间的算法实现了 PDF 阅读器 (Electron 应用) 非连续文本选中和高亮显示的功能.
\item 通过使用正则表达式提取目录页文本中标题和页码的方法, 实现了自动生成 PDF 章节跳转链接的功能.
\end{itemize}

\textbf{前端开发实习生 | PegasusCRM | 亚特兰大} \hfill 2018/01 -- 2018/04
\begin{itemize}
\item 负责用 Vue.js 重写原先基于 jQuery 插件的表格组件并实现拖拽分组的功能, 同时参与维护和更新旧代码.
\item 使用 Laravel Blade 模板和 SCSS 复现页面设计和样式规范, 参与系统设计讨论并协助后端调试代码.
\end{itemize} 


\section*{\texttt{\large //github.com/GalMunGral}}
\textbf{ \texttt{/replay:} 自主开发的类 React + MobX  前端框架} \hfill 2020/05 -- 2020/09
\begin{itemize}
\item 独立设计并实现了 diff 算法, 并通过 generator 实现了可中断, 可取消并且支持组件懒加载的非阻塞渲染模式.
\item 实现了基于 proxy 的响应式机制, 并在此基础上实现了仿 Redux + Vuex 的状态管理和简单的前端路由.
\item 编写了基于 Babel 的 webpack loader, 实现了对 JSX 以及一套原创的基于 JS 自身语法的 DSL 的支持.
\item 使用此框架搭建了仿styled-components的CSS-in-JS库和仿 Gmail 的单页应用以作测试/调试用.
\end{itemize}

\textbf{\texttt{/open-evaluation:} 期中调查问卷系统 (设计课程项目)} \hfill 2019/08 -- 2019/12
\begin{itemize}
\item 参与了 Angular 前端开发, 完成了将 JSON 格式的问卷模板和收集的用户回答动态渲染成表单的功能.
\item 使用 Node.js 集成了学校单点登录 (CAS) 系统, 并通过 Canvas LMS API 拉取学生选课数据.
\item 用 Puppeteer 模拟用户登录并提取 cookie 作为临时替代 OAuth 2.0 的获取 API 访问权限的方法 (仅作 demo 用).
\end{itemize}

\textbf{\texttt{/marta-\{react,vue\}:} 地铁乘客信息管理系统 (SQL课程项目) } \hfill 2017/09 -- 2017/11
\begin{itemize}
\item 使用 MySQL, Express 和 JWT 搭建了后端 API, 提供记录行程, 管理账户余额, 查看历史记录和车站人流等功能.
\item 分别使用 React/Redux/React Router 和 Vue/Vuex/Vue Router 搭配 Bulma  CSS 框架开发了两套前端界面.
\item 使用 Node.js 编写了以 cookie 指定上游端口的反向代理, 实现了在同一个 Heroku 项目同时部署多个应用.
\end{itemize}

\textbf{\texttt{/web-repl:} 在线 Python REPL 聊天室}  \hfill 2018/05 
\begin{itemize}
\item 通过动态创建 Python 子进程的方式, 用 Node.js 和 Socket.IO实现了可供多个用户共同运行代码的简易聊天室.
\item 在已有的 Socket.IO 通信的基础上, 为 WebRTC 搭建了信令服务, 并实现了视频通话的功能.
\item 为替代 Socket.IO 使用, 使用 Node.js 依照协议规范编写了简易的 WebSocket 服务器.
\end{itemize}

\textbf{\texttt{/js-to-wasm:}简易 JavaScript -- WebAssembly 编译器 (仅支持小部分语法)} \hfill 2020/06
\begin{itemize}
\item 使用 Node.js 开发了大致遵循 ESTree 规范 (解析器部分) 并美观输出 \texttt{.wat} 文件的 JS 编译器.
\end{itemize}

\textbf{\texttt{/sitbit-\{ios,android,server\}:} 移动端久坐追踪应用} \hfill 2019/01
\begin{itemize}
\item 使用 iOS Core Motion 和安卓 sensor 框架实时监测起身和坐下的动作, 并上传数据到用后端 (用Go语言搭建)
\item 使用 D3.js 和 SVG 以类似 GitHub 个人主页上的日历热图的形式展示每日久坐时间.
\end{itemize}

\textbf{\texttt{/todo-\{uikit-objc,swiftui,flutter,react-native,nativescript\}:} 移动端To-do List} \hfill 2019/06

\end{document}

