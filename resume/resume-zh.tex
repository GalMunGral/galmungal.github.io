\documentclass[10.5pt]{article}
\usepackage{xeCJK}
\usepackage[margin=0.5in]{geometry}
\usepackage{titlesec}
\usepackage{enumitem}
\usepackage[T1]{fontenc}

\titleformat{\section}{}{\thesection}{0pt}{\vspace{-0.5em}}[{\titlerule[0.2pt]}]
\setCJKmainfont{Noto Serif CJK SC}
\setlist[itemize]{leftmargin=1em, topsep=0.6em, itemsep=0pt, label={-}}
\setlength\parindent{0pt}
\pagestyle{empty}

\begin{document}

\parbox{0.25\textwidth}{\hfill}
\parbox{0.5\textwidth}{\centering\huge\bfseries 何文琦}
\parbox{0.25\textwidth}{\texttt{github.com/GalMunGral\\hewenqi96@qq.com\\ 136-2602-2079 }}

\section*{教育经历}
\textbf{计算机科学理学学士,佐治亚理工学院 (Georgia Institute of Technology)} GPA: 3.97/4.0 \hfill 2015/08 - 2019/12

\section*{工作经历}
\textbf{兼职Electron开发工程师, Etude 阅读器 (大学生初创公司),亚特兰大} \hfill 2019/09 - 2019/12
\begin{itemize}
\item 应用MVVM模式并使用插入区间的算法实现了非连续文本选中 (高亮显示) 和撤销选中的功能.
\item 通过利用正则表达式解析PDF目录页文本中的标题和页码, 实现了自动生成各章节跳转链接的功能.
\end{itemize}

\textbf{前端开发实习生, PegasusCRM, 亚特兰大} \hfill 2018/01 - 2018/04
\begin{itemize}
\item  参与系统设计讨论和前端开发, 使用Laravel Blade模板和 SCSS 复现页面设计稿和总体设计规范.
\item 使用 Vue.js 重新实现了表格组件并替代了原本的 DataTables jQuery 插件, 以满足不断出现的新功能需求.
\item 利用 HTML5 拖放 API 实现了通过拖拽对客户信息进行分组的功能.
\end{itemize}


\section*{学校/个人项目}
\textbf{Koala: 类React + MobX + styled-components 前端框架} \hfill 2020/04 - 2020/05
\begin{itemize}
\item 实现了依赖收集, 组件异步更新, 非阻塞渲染, 组件懒加载, DOM 批量操作等特性, 并模拟动态作用域从而实现隔代通信.
\item 编写了简单的 Babel 插件, 在JavaScript 语法内创建了基于函数嵌套的 DSL 作为 JSX 和模板的替代.
\item 使用此框架开发了仿Gmail的单页应用, 同时实现了简易的前端路由以及类似 Redux 的状态管理和时间旅行功能.
\end{itemize}


\textbf{简易JavaScript-WebAssembly 编译器} \hfill 2020/04, 2020/06
\begin{itemize}
\item 使用 Node.js 开发了简易的 JavaScript 编译器 (仅支持部分语法), 可输出美观打印的 WebAssembly 文本格式.
\item 实现了大致遵循 ESTree 规范的 shift-reduce 解析器 (仅支持部分语法) 作为 Babel parser 的替代 .
\end{itemize}

\textbf{MARTA 公共交通管理系统 (SQL课程项目) } \hfill 2017/09 - 2017/11
\begin{itemize}
\item 使用 MySQL, Express 和 JWT 搭建了后端 API, 支持记录行程, 管理余额, 查看历史记录和车站人流等信息.
\item 使用 React/Redux/React Router 和 Vue/Vuex/Vue Router 搭配 Bulma CSS 框架开发了两套前端单页应用.
\item 通过利用 cookie 指定上游服务器端口号的方法, 用 Node.js 实现了简易的可配置的反向代理.
\end{itemize}

\textbf{佐治亚理工期中调查问卷系统 (设计课程项目)}  \hfill 2019/08 - 2019/12
\begin{itemize}
\item 参与了 Angular 前端开发, 实现了将 JSON 格式的问卷模板和用户回答动态渲染成表单的功能.
\item 使用 Node.js 集成了 Georgia Tech Login Service (学校官方单点登录系统) 并利用 Canvas LMS API 拉取学生选课数据.
\item 在集成 OAuth 2.0 前, 作为 demo 的临时方案, 利用了 Puppeteer 在模拟用户登录后提取 cookie 以获取 API 访问权限.
\end{itemize}

\textbf{在线 Python REPL 聊天室}  \hfill 2018/05, 2020/3
\begin{itemize}
\item 使用 Node.js 通过动态创建可与多个客户端交互的 Python REPL 子进程的方式搭建了简易的聊天室.
\item 使用 Socket.IO 广播用户消息 (REPL 的输入和输出) 并为 WebRTC 视频通话的功能提供信令服务.
\item 使用 Node.js 实现了简易的 WebSocket 服务器作为 Socket.IO 的替代.
\end{itemize}

\textbf{仿 Visual Studio Code}  \hfill 2019/05
\begin{itemize}
\item 使用 Electron, React 和 styled-components 仿制了简易的桌面文件浏览器和文本编辑器. 
\end{itemize}

\textbf{SitBit: 移动端久坐时间追踪应用} (PoC) \hfill 2019/01
\begin{itemize}
\item 使用 iOS Core Motion 框架和 Android 传感器框架在后台监测起身和坐下的动作.
\item 使用 D3.js 和 SVG 以类似 GitHub 个人主页上的日历热图的形式展示每日久坐时间.
\end{itemize}

\textbf{移动端 To-do 应用} (PoC) \hfill 2019/06
\begin{itemize}
\item 分别使用 UIKit (Objective-C), SwiftUI, Flutter, React Native, NativeScript 实现了简单的To-do应用.
\end{itemize}

\end{document}

